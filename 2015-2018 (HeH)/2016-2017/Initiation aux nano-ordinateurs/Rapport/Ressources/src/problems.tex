\section{Problèmes rencontrés}
\label{sec:probl-renc}

\subsection{Joystick}
\label{sec:joystick}

La librairie de python comportait une erreur dans la prise en charge du joystick
\footnote{\href{https://www.element14.com/community/community/raspberry-pi/raspberry-pi-accessories/blog/2017/01/23/raspberry-pi-sense-hat-enabling-the-joystick}{https://www.element14.com/}}. Il
a fallu supprimer les fichiers d'extension \og \textit{.py} \fg relatifs au
Sense HAT et télécharger les fichiers mis à jour.

Pour terminer, il était nécessaire de supprimer un point mal placé dans l'une des
importations de librairies.

\subsection{Matrice}
\label{sec:matrice}

Étant donné la taille du Raspberry Pi, la matrice du Sense HAT s'est avérée fort petite
pour l'implémentation du jeu de Pong traditionnel.

De ce fait, il y avait manque d'espace pour la taille des raquettes que nous
avons été obligé de limiter à une LED, tout comme pour la balle, ce qui augmente
la difficulté de prise en main du jeu.

De plus, les angles ont aussi été atteints par cette limitation matricielle, nous
empêchant d'utiliser d'autres angles que des multiples de 45.

\subsection{Gyroscope}
\label{sec:gyroscope}

Dans le cadre du projet, il nous a été demandé d'ajouter, comme fonctionnalité
supplémentaire, une version du jeu se basant sur l'utilisation du gyroscope
incorporé au Sense HAT.

Suite à un manque d'ergonomie lié à la manipulation du joystick et de
l'inclinaison du gyroscope, nous avons décidé d'implémenter une intelligence
artificielle (IA) comme autre fonctionnalité.

%%% Local Variables:
%%% mode: latex
%%% TeX-master: t
%%% End:
