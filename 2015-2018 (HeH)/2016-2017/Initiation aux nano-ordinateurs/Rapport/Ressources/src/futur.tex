\section{Possibilités futures}
\label{sec:poss-futures}

Le code étant implémenté de manière soignée et en orienté objet, celui-ci offre
beaucoup plus de possibilités pour d'éventuelles évolutions.

Lors de la conception de la version préliminaire du jeu, nous avons prévu plusieurs
évolutions possibles pour notre Pong dans le cas où nous aurions plus de temps.

En voici quelques unes:
\begin{itemize}
	\item le mouvement des raquettes sur deux axes: en plus de se déplacer sur
	l'axe vertical, les raquettes se seraient déplacées à l'horizontal.
	\item la taille des raquettes: plus le niveau serait haut, plus la raquette rapetisserait.
	\item l'ajout d'obstacles sur le terrain: des murs seraient placés de manière
	aléatoire sur le terrain pour augmenter la difficulté du jeu.
	\item les angles: avec une taille de raquette plus importante, l'implémentation de
	différents angles (50$^{\circ}$, 70$^{\circ}$, etc.) rendrait le jeu plus intéressant.
	\item la difficulté de l'IA: une seconde IA plus efficace se replacerait au centre de
	son axe dès que la balle rebondirait sur sa raquette.
\end{itemize}

%%% Local Variables:
%%% mode: latex
%%% TeX-master: t
%%% End:
