\section{Conclusion}
\subsection{Futur}
Selon les statistiques, les parts de marché de Windows sont en baisse suite à
l’augmentation de l’utilisation des périphériques mobiles (smartphones et
tablettes) dont les systèmes d'exploitation sont principalement Android et iOS.
Ce qui annonce le possible remplacement des ordinateurs par les smartphones et
tablettes de plus en plus puissantes. \\

Néanmoins, Microsoft a déjà annoncé son futur système d'exploitation
\textit{Midori} qui provient de \textit{Singularity}, un projet expérimental
sorti en 2003, et qui ne s'installerait pas sur le disque dur, comme les
anciens systèmes d'exploitation l'étaient, mais dans une interface Web. \\

Malgré cela, plusieurs avis divergent à propos de l'avenir de l'OS. Concernant
un futur proche, certaines personnes pensent que l'avenir se trouve du côté
du \textit{cloud computing} offrant un espace client avec des composants clés
pendant que les systèmes secondaires seront diffusés, chargés et enlevés au
besoin. Un tel système nécessitera fiabilité et une quasi constante accessibilité
des services de haute vitesse. \\
Néanmoins, il faudrait y apporter diverses modifications pour le coût de la
bande passante. \\

D'autres estiment que le futur réside dans l'association de GNU/Linux et
\textit{Docker} (une plateforme ouverte pour les applications distribuées aux
développeurs et aux administrateurs système). \\

Ensuite, pour un futur à plus long terme, nous nous dirigeons vraisemblablement
vers les ordinateurs quantiques, l'idée de l'OS sera alors radicalement différente.
De ce fait, le calcul sera fait dans de nombreux appareils, tous synchronisés
ensemble et les périphériques d'entrées/sorties correspondront à chaque
surface plane, comme des claviers projetables sensibles à l'infrarouge. Par
conséquent, dans un tel environnement informatique, un système d'exploitation à
base de dispositifs uniques sera obsolète. \\

Ce que l'on peut retenir globalement, c'est qu'il est fort probable que les futurs
systèmes d'exploitation ne seront plus stockés dans le disque dur, mais sur un
disque dur virtuel ou sur un serveur distant. \\
C'est la direction que prend \textit{Chrome OS} qui est un système connecté
s’appuyant  sur des applications web.

\newpage

\subsection{Choisir son OS}
Choisir son système d'exploitation n'est pas une mince affaire et cela
doit être un choix judicieux puisqu'il est envisagé sur du long terme. Par
conséquent, ce choix dépend des besoins de l'utilisateur. \\
Néanmoins, les avantages et les inconvénients à retenir des différents systèmes
d'exploitation sont les suivants : \\

\textbf{Windows :}

\begin{enumerate}
\item Avantages : \\

  \begin{itemize}
  \item Compatible avec la plupart des logiciels et jeux vidéo; \\

  \item Dispose d'une très grande communauté en cas de problème; \\

  \item Interface graphique conviviale et simple d'utilisation. \\
  \end{itemize}

\item Inconvénients : \\
  \begin{itemize}
  \item Cible de la majorité des virus (\textit{nécessité d'un
antivirus et d'un pare-feu}); \\

  \item Toutes informations est sauvegardée par \textit{Microsoft}; \\

  \item Licence payante $\sim$ 150 \euro  (\textit{souvent comprise dans l'achat de l'ordinateur}).
\\
  \end{itemize}
\end{enumerate}

$\Rightarrow$ Windows sera privilégié pour les jeux vidéo et pour les
utilisateurs ne possédant pas d'exigences particulières en informatique.
Il le sera aussi pour les utilisateurs recherchant la plus grande comptabilité,
étant donné que \\
95 \% des écoles et des bureaux sont maintenant équipés de PC. \\
De plus, le prix dépend du large choix de modèles disponibles sur le marché. \\

\newpage

\textbf{GNU/Linux :}

\begin{enumerate}
\item Avantages : \\

  \begin{itemize}
  \item Très léger; \\

  \item Open source et personnalisable; \\

  \item Ne nécessite pas d'antivirus, mais un pare-feu est requis. \\
  \end{itemize}

\item Inconvénients : \\

  \begin{itemize}
  \item Faible compatibilité avec les logiciels \textit{Microsoft} et peu de jeux
vidéo; \\

  \item Peut s'avérer complexe au premier abord; \\

  \item La plupart des aides sont disponibles en anglais. \\
  \end{itemize}
\end{enumerate}

$\Rightarrow$ GNU/Linux est avant tout une mentalité, une façon de penser.
Il sera privilégié par les utilisateurs avertis faisant abstraction des jeux
vidéo et ayant des exigences par rapport à la rapidité et la stabilité de leur
système. Ou ayant des exigences quant à la liberté: GNU/Linux étant libre,
l'utilisateur peut installer et fabriquer sa propre distribution sans problème.  \\

\clearpage

\textbf{Mac OS X :}

\begin{enumerate}
\item Avantages : \\

  \begin{itemize}
  \item Stable, fiable et rapide; \\

  \item Interface graphique conviviale; \\

  \item Ne possède que très peu de virus à ce jour (/textit{un antivirus n'est
  pas obligatoire, mais mieux vaut prévenir que guérir}).\\
  \end{itemize}

\item Inconvénients : \\

  \begin{itemize}
  \item Faible bibliothèque de jeux vidéo; \\

  \item Coûteux; \\

  \item Peu d'aide en français, mieux vaut se tourner vers l'anglais. \\
  \end{itemize}
\end{enumerate}

$\Rightarrow$ Mac OS X sera privilégié pour un utilisateur possédant une
exigence envers le design, l'interface graphique ainsi que la rapidité et
faisant abstraction du prix et de la faible offre du côté des jeux vidéo. \\
Il sera aussi privilégié par tout développeur cherchant à coder une application
iOS ou cherchant la puissance du terminal d'UNIX sans passer par Linux.

\newpage

\subsection{Résultat}
Un système d'exploitation est donc un super-logiciel sans lequel aucun ordinateur
ne fonctionnerait (\textit{de l'ordinateur à la voiture, en passant par les smartphones
et les fours à micro-ondes}). \\
En effet, sans lui, pas de démarrage, de fonctionnement de logiciels ou de
gestionnaire de fichiers. C'est lui qui gère les ressources matérielles et la
fourniture de services aux applications. \\
Celui-ci est l’intermédiaire entre les logiciels, l’utilisateur et le matériel. \\

Il a de nombreux rôles comme la gestion des entrées/sorties (\textit{claviers,
souris, imprimantes, disques durs externes, etc.}), la gestion des fichiers,
des droits (\textit{restrictions de l'accès aux fichiers critiques}), de la
sécurité ou encore des environnements de bureau. \\

Il est à noter que pour certains de ces appareils - par exemple les fours à
micro-ondes - le nom est \textit{système embarqué}. \\

Vu qu'il existe d'innombrables dispositifs électroniques ayant besoin d'un OS,
il y a de nombreux systèmes d'exploitation différents. \\
Pour les GSM et smartphones, il y avait Symbian, Meego ou encore Bada; maintenant
principalement remplacés par Android, iOS et Windows Phone. \\
Pour les ordinateurs, Windows est le plus utilisé dans le monde loin devant
Mac OS X, GNU/Linux ou encore Chrome OS. \\
Du côté des serveurs, ce sont les distributions Linux qui sont les plus populaires
(\textit{Red Hat, SuSE, Ubuntu,…}) avec 98,17 \% des serveurs web sous Linux. \\
De même pour les superordinateurs (ou \textit{supercalculateurs}) où 92,4 \% des
500 plus puissants tournent sous Linux et 5 \% sont sous UNIX. \\

Lors des dernières années, les ordinateurs ont connu une baisse de ventes au
profit des smartphones et tablettes de plus en plus puissants. \\
C'est pourquoi les grandes entreprises développent leur système mobile et qu'elles
rapprochent, chacune à sa manière, leurs systèmes d'exploitation pour mobiles
et ordinateurs.
