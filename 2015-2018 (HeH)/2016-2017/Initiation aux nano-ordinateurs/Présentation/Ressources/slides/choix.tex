\section{Choix}
    
    \begin{frame}
        
        \begin{center}
            \huge{Choix}
        \end{center}
        
    \end{frame}
    
    \begin{frame}
        \frametitle{Langage}
        
        Notre choix s’est naturellement porté vers le \textbf{Python} :
        
        \begin{itemize}
        
            \item c'est le langage de prédilection sur le Raspberry Pi;
            \item il dispose d’une documentation et d’une vaste communauté. \\
        
        \end{itemize}
        
        ~\\
        En outre, une librairie permettant d’interagir avec Sense HAT est disponible en Python.
        
    \end{frame}
    
    \begin{frame}
        \frametitle{Système d'exploitation}
        
        Pour notre utilisation, \textbf{Raspbian}, étant le choix suggéré par le constructeur, a été choisi.
        
        \begin{figure}[!h]
            \centering
            \includegraphics[scale=0.06]
            {images/choix/raspbian.pdf}
            \caption{Logo Raspbian}
        \end{figure}
        
    \end{frame}
    
    \begin{frame}
        \frametitle{Outils de développement}
        
        Pour leur simplicité ainsi que pour notre familiarité avec eux, nous avons utilisé GNU Emacs et Atom. \\
        De plus, nous avons utilisé GitHub  afin de synchroniser nos avancées.
        ~\\
        ~\\

        \begin{figure}[!h]
            \centering
            \begin{minipage}[c]{0.25\textwidth}
                \centering
                \includegraphics[scale=0.06]
                {images/choix/emacs.pdf}
                \caption{Logo Emacs}
            \end{minipage} \qquad
            \begin{minipage}[c]{0.25\textwidth}
                \centering
                \includegraphics[scale=0.06]
                {images/choix/atom.pdf}
                \caption{Logo Atom}
            \end{minipage} \qquad
            \begin{minipage}[c]{0.25\textwidth}
                \centering
                \includegraphics[scale=0.06]
                {images/choix/github.pdf}
                \caption{Logo GitHub}
            \end{minipage}
        \end{figure}
        
    \end{frame}