\section*{Questionnaire}
\addcontentsline{toc}{section}{Questionnaire}

\begin{enumerate}
  \item Guillaume est dubitatif lors de la lecture d’un site sur lequel est écrit: \\
  « La ligne de commande est apparue pour faciliter le dialogue entre les OS
  et les utilisateurs, vu que l’interface graphique n’était pas assez
  puissante ». \\
  Le Webmaster a-t-il raison ? \textbf{Faux} \\

  Depuis les années 80, l’interface graphique est utilisée à la place de
  l’interface en ligne de commande, car elle ne nécessite pas l'apprentissage
  de commandes pour utiliser un ordinateur.\\
  Par contre, sur les systèmes dérivés d’UNIX, la ligne de commande reste
  fortement utilisée étant donné la richesse des possibilités. \\

  \item D’après Gino, plus de 90 \% des 500 ordinateurs les plus puissants au monde
  et des serveurs seraient équipés de Linux. \\
  Est-ce bien vrai ? \textbf{Vrai}\\

  En effet, Linux tourne sur 98 \% des serveurs Web mondiaux et \\
  sur 92 \% des 500 supercalculateurs les plus puissants. \\

  \item Lors d’une présentation, Alexandre annonce que les systèmes d’exploitation
  d’Apple sont basés sur Ubuntu, mais cette affirmation ne fait pas
  l’unanimité. \\
  A-t-il raison ? \textbf{Faux} \\

  Mac OS X et ses dérivés sont basés sur \textit{NeXTSTEP} qui l'était lui-même
  sur l’implémentation BSD d’UNIX. \\
  Ubuntu est basé sur \textit{Debian} qui est un UNIX-like. \\

  \item Un site affirme que la version de Windows 10 pour tablettes est identique à
  celle pour smartphones, mais Antoine n’y croit pas. \\
  Le site a-t-il raison ? \textbf{Faux} \\

  Le système d'exploitation des \textit{Surfaces} de Microsoft est Windows 10,
  alors que les Windows Phones utilisent un système adapté: Windows 10 Mobile. \\

  \clearpage

  \item Lors de son entrevue avec les policiers, Julien soutient que sa voiture a
  été piratée sur l’autoroute avant qu’elle ne s’éteigne d’un coup. \\
  Est-ce possible ? \textbf{Vrai} \\

  Les voitures étant contrôlées par un ordinateur de bord, il suffit que la
  voiture soit connectée à Internet pour qu'un hacker la pirate. \\
  Une fois piratée, il peut très bien arrêter le moteur, bloquer les freins ou
  même prendre la main sur le volant ! \\
  Tout cela à distance, bien entendu. \\

  \item Sarah a installé Ubuntu en Dual Boot avec Windows. Après avoir installé des
  logiciels téléchargés depuis Linux, sa session Windows est infectée par un
  virus. \\
  Est-ce possible d’attraper un virus depuis Linux ? \textbf{Vrai} \\

  Il est en effet possible de télécharger des virus pour n'importe quel OS
  depuis Internet. Une fois le virus téléchargé, l'ordinateur devient un
  porteur sain si le virus n'est pas prévu pour ce système. \\
  Il est donc concevable qu'un fichier inoffensif sur Linux soit transféré sur
  un autre OS qui, lui, devienne infecté. \\
  C'est pour cela qu'il est important d'avoir un pare-feu, antivirus
  (\textit{qui bloquera les virus pour tous les OS}), ... \\

  \item Lors d’un labo d’électronique, Thibault demande à Alexandre de lui envoyer
  une photo de son circuit. \\
  Ayant tous deux un iPhone, ils se l’échangent grâce à Quick Look. \\
  Ont-ils utilisé la bonne fonctionnalité ? \textbf{Faux} \\

  Quick Look permet de montrer l’aperçu d’un fichier, dossier ou site ou la
  définition d’un mot. \\
  Pour le partage de fichiers, c’est AirDrop qui est utilisé, à condition
  d’activer le Bluetooth et d’être sur le même réseau Wi-Fi. \\

  \clearpage

  \item Lorsqu’il se relit, Clément pense s’être trompé en notant que « un
  environnement de bureau est un ensemble de logiciels et d’un système
  d’exploitation, comme Ubuntu et Red Hat ». \\
  Cette note est-elle correcte ? \textbf{Faux} \\

  Cette définition est celle d'une distribution. \\
  Un environnement de bureau, lui, constitue les caractéristiques graphiques
  du système d’exploitation et permet à l’utilisateur d’interagir avec son
  ordinateur. \\
  C'est lui qui gère les fenêtres, le bureau, etc. \\

  \item Lorenzo ne croit pas qu’Android est basé sur Linux et qu’il peut être
  utilisé sur ordinateur. \\
  L’information est-elle correcte ? \textbf{Vrai} \\

  Android est basé sur un noyau Linux. \\
  Initialement disponible uniquement sur smartphones, Android s'est décliné sur
  tablettes, télévisions, consoles de jeux et montres connectées. \\
  Depuis 2011, Android-x86 est disponible pour les ordinateurs possédant un
  processeur x86 et x64 de Intel. \\

  \item Arnaud ne croit pas qu’un OS a deux fonctions majeures: la gestion des
  ressources matérielles et la fourniture de services aux applications. \\
  Thomas lui rétorque qu’il se trompe, mais est-ce vrai ? \textbf{Vrai} \\

  En effet, un système d'exploitation a deux fonctions majeures. \\
  Si une application requiert des informations, c'est à lui qu'elle fait appel. \\
  Elle s'occupe aussi du démarrage, de la gestion du processeur et de la
  mémoire. \\

  \item Avant de l’appeler iOS, Steve Jobs l’a nommé \textit{idealOS} vu qu’il était \\
  « le système d’exploitation idéal pour l’iPhone ». \\
  Cette information venant de Wikipédia est-elle véridique ? \textbf{Faux} \\

  C'était le système d'exploitation des iPhones ainsi que des iPod Touch.
  Il était alors appelé \textit{iPhoneOS}. \\
  En 2010, il devient iOS avec la sortie de l'iPad. \\

  \item Lors d’une présentation, Laurent entend cette affirmation : les OS ont de
  nombreux rôles comme la gestion des droits, de la sécurité ainsi que
  l’exécution des logiciels. \\
  Est-elle vraie ? \textbf{Vrai} \\

  Un système d'exploitation a de nombreux rôles dont la gestion des droits, de
  la sécurité et l’exécution des logiciels. \\
  Cependant, on peut rajouter la gestion du processeur, de la mémoire ainsi que des
  environnements de bureaux. \\

  \item Pendant son stage, Anthony a appris, avec scepticisme, que Cisco IOS était
  basé sur Linux. \\
  Après quelques recherches, il est fixé, mais était-ce vrai ? \textbf{Faux} \\

  Cisco IOS n'est pas lié à Linux: c'est un système développé entièrement par
  Cisco Systems. \\
  Par contre, Cisco IOS XE (ou IOS XE) est bien basé sur Linux et est
  entièrement compatible avec Cisco IOS. \\

  \item Florian prétend qu’il peut déverrouiller son ordinateur sous Windows avec
  ses yeux, ses amis ne le croient pas. \\
  Dit-il la vérité ? \textbf{Vrai} \\

  Windows 10 intègre Windows Hello, qui permet de déverrouiller sa session avec
  son visage, son iris ou son doigt à condition de posséder un périphérique
  compatible. \\

  \item D’après les dires de Burak, il est tout à fait possible d’utiliser Windows
  10 Mobile comme on le fait avec un ordinateur. \\
  Est-ce vrai ? \textbf{Vrai… et Faux} \\

  Grâce à une station d'accueil dédiée, on peut utiliser son smartphone sous
  Windows 10 Mobile sur un grand écran avec un clavier et une souris. \\
  Mais ce n'est que Windows Mobile, il ne peut pas faire autant qu'un
  ordinateur: moins de fonctionnalités, pas de multitâche, etc.

\end{enumerate}
