\section{Présentation du projet}
\label{sec:presentation}


\subsection{Introduction}
\label{subsec:intro}
En cette troisième et dernière année de bachelier en Informatique \& systèmes, le temps est venu d'effectuer le fameux TFE.\\
Le TFE, \textit{ou Travail de Fin d'Études}, est l'aboutissement de  trois années d'apprentissage, tant théorique que pratique, qui permet à l'étudiant de révéler les connaissances et le savoir-faire qu'il a acquis dans le domaine de son cursus.


\subsection{Le projet}
\label{subsec:projet}
Le site permettra d'apprendre et de créer des cours de tous niveaux, et cela, sur n'importe quel sujet (langue, informatique, sciences,...).

Le site proposera, en tant qu'exemples, différents cours composés de théorie et d'exercices. \\
Il fournira aussi un outil de création de cours afin que les utilisateurs puissent proposer de nouveaux sujets. \\

L'apprentissage se fera en trois étapes :
\begin{enumerate}

    \item L'utilisateur découvrira le nouveau sujet par de la théorie ainsi que par un ou plusieurs exemples. Il en apprendra alors l'utilité et le fonctionnement.
    
    \item Entre deux parties théoriques, l'utilisateur mettra en pratique ce qu'il aura appris au travers de petits QCM.
    
    \item Une fois le chapitre terminé, un questionnaire (QCM, ordonnancement du code,...) sera proposé à l'utilisateur. Celui-ci sera noté sur 10 afin que l'utilisateur puisse se juger et s'améliorer. \\
    Le passage au chapitre suivant requerra une cote minimale de 7/10. \\
    
\end{enumerate}

Il pourra être utilisé aussi bien par les écoles que par \og \textit{les particuliers} \fg.

D'autre part, celui-ci est la suite du projet que j'ai réalisé en deuxième bachelier dans le cadre du cours de \textit{Gestion de projets}.

\newpage

\subsection{Fonctionnalités}
\label{subsec:fonctionnalites}
Voici les différentes fonctionnalités à implémenter dans le site :

\begin{itemize}

    \item[--] Apprentissage du cours
    
    \item[--] Exercices pour fixer la matière
    
    \item[--] Trophées de réussite de chapitres et du cours
    
    \item[--] Statistiques pour chaque étudiant ainsi que pour le créateur du cours
    
    \item[--] Enregistrement depuis un fichier \textbf{.csv} et envoi automatique de mails dans ce cas
    
    \item[--] Différents QCM disponibles, dont certains seront choisis au hasard (\textit{aussi par l'import d'un fichier})
    
    \item[--] Possibilité de cacher des questions et de générer un PDF depuis toutes les questions
    
    \item[--] Possibilité de générer un questionnaire d'après un/des chapitre(s) choisi(s)
    
    \item[--] Correction des QCM personnalisée d'après la réponse \textit{(semi-automatique)}
    
    \item[--] Différents niveaux de difficulté d'après l'âge ou le niveau de l'utilisateur
    
    
    \item[--] \textit{Bonus}
    
    \begin{itemize}
    
        \item[--] Choix de la difficulté des questions, ainsi que gestion de leur pondération
        
        \item[--] Création d'une application mobile \textit{Android} pour un accès hors-ligne
        
    \end{itemize}
    
\end{itemize}