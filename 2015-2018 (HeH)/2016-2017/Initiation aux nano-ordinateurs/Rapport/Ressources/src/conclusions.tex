\section{Conclusion}
\label{sec:conclusion}

\subsection{Générale}
\label{sec:generale}

Ce projet nous a permis d'apprendre à utiliser le Raspberry Pi, le Sense HAT
ainsi que les nouveautés de Python 3 que nous n'avions pas encore eu la chance
d'essayer. \\
En outre, nous avons pu mieux comprendre le comportement du Sense HAT tel que
vu au cours.

Nous avons donc implémenté un jeu de Pong en Python pour le Sense HAT, en tirant
profit du joystick qui était à disposition.\\
Le jeu nous semblant \og \textit{fade} \fg, il nous est venu à l'idée de lui
rajouter un mode contre une intelligence artificielle.

Grâce à cela, nos connaissances en Python et en algorithmique ont été renforcées. \\
La prise en charge de collisions a été un challenge et un plus pour notre
expérience, car totalement inédit dans nos précédents projets. \\

Ayant perdu 8 heures sur les 20 initialement prévues à cause du manque d'énoncé
et de matériel lors des premières séances, nous avons dû nous dépasser pour
terminer le projet dans les délais. \\
Cela nous a ainsi permis de nous améliorer dans notre gestion du temps.

\newpage

\subsection{Conclusions individuelles}
\label{sec:individuelles}

 \begin{fquote}[Terencio Agozzino]Me concernant, j'ai déjà pu apprivoiser le
   langage Python à l'Université lors d'anciens projets, ainsi que pour des
   projets personnels. Néanmoins, ce projet m'a permis de manipuler un Raspberry
   Pi en fonction de l'algorithmique et prendre conscience de la complexité de
   celui-ci avec le hardware.
 \end{fquote}

 \begin{fquote}[Alexandre Ducobu]Grâce à ce projet, je me suis perfectionné en
 algorithmique par la prise en charge des angles liés aux collisions.\\
 Il y a un an, nous avions eu l'occasion de programmer un pic en C.\\
 Ce projet était ainsi une continuation m'ayant permis d'utiliser
 un Raspberry Pi, chose qui me tentait depuis plusieurs années.
 \end{fquote}

%%% Local Variables:
%%% mode: latex
%%% TeX-master: t
%%% End:
