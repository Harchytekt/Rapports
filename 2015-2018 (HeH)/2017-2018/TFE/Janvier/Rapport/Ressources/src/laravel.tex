\section{Fonctionnement de Laravel}
\label{sec:laravel_functioning}


\subsection{Authentification}
\label{subsec:auth}
L'authentification est nativement gérée par Laravel \textit{(au travers du module} auth\textit{)}.

Une fois activée, il faut migrer la base de données afin de prendre en compte l'authentification.\\
Les utilisateurs peuvent désormais s'inscrire, se \textit{(dé)}connecter ou réinitialiser leur mot de passe \textit{(une fois un serveur mail ajouté)}.

\subsubsection{Les middlewares}
\label{sec:middlewares}
Les pages sont protégées par des middlewares \textit{(Authenticate.php} et \textit{RedirectIfAuthenticated.php)}. \\
Les middlewares forment une couche de protection entre les requêtes et l'application : ils effectuent des traitements à l'arrivée ou au départ des requêtes. \\
Par exemple, la gestion des sessions, des cookies et de l'authentification se fait dans un middleware. On a plusieurs couches de middlewares \textit{(comme des pelures d'ognon)}. Chacun effectue son traitement et transmet la requête ou la réponse au suivant.

Le premier, \textit{Authenticate.php}, vérifie qu'un utilisateur est authentifié. \\  
Si ce n'est qu'un invité \textit{(guest)}, alors on renvoie \textit{Unauthorized} si la requête est en Ajax. Si elle n'est pas en Ajax, on fait une redirection vers la page d'authentification \textit{(route} login\textit{)}. \\
Si ce n'est pas un invité, on exécute la requête.

Le second, \textit{RedirectIfAuthenticated.php}, fonctionne pratiquement à l'opposé du fichier \textit{Authenticate.php}. \\
Si l'utilisateur est authentifié \textit{(auth)}, alors on le renvoie à la page de base du site. \\
Sinon, on exécute la requête.

La majorité des pages du site ne doivent être accessibles qu'aux utilisateurs authentifiés.
C'est là qu'entrent en jeu les middlewares : il suffit d'ajouter \textit{middleware('auth')} à la route ou au contrôleur d'une page pour en interdire l'accès aux invités. \\
C'est donc \textit{Authenticate.php} qui est utilisé.


\subsection{Migration}
\label{subsec:migration}
Laravel propose cet outil dont le but est de s'occuper de la maintenance de la base de données de l'application.\\
Une migration permet donc de créer et de mettre à jour une base de données en créant et supprimant des tables, des colonnes dans ses tables, en créant des index...

\newpage
\subsection{Routage}
\label{subsec:routage}
Le rôle principal du routage est de faire correspondre un \textit{nommage particulier} représentant un accès aux services de l'application vers le contrôleur associé.\\
Une fois le nommage effectué, le serveur web doit faire correspondre les adresses aux bonnes actions.\\
Par exemple, l'adresse \textit{http://learnit.dev/authenticated/mesCours/inscrits}
devient \textit{http://learnit.dev/coursinscrits}.

