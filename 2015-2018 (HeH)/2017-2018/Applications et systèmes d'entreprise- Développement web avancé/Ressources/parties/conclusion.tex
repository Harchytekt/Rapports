%!TEX root = ../apese-rapport.tex

\section{Conclusion}
Pour conclure, ce projet nous a permis de découvrir le monde du développement en entreprise tout en restant à l'école. Le fait de travailler en équipe tout en utilisant un gestionnaire de versions et des technologies utilisées nous prépare efficacement à travailler dès la fin de cette année. \\
Toucher à toutes les étapes de ce projet (front-end, back-end, organisation, \dots), nous montre ce qui nous attend afin de mieux nous former. \\

La plateforme web de prise de rendez-vous est accessible depuis une page de connexion des secrétaires. Une fois la connexion effectuée, l'accès au site est sécurisé par un token. \\
Les secrétaires peuvent naviguer entre différents onglets afin d'afficher les données provenant de la base de données : les docteurs, les clients, les salles, \dots \\
%?? Bien entendu, il est possible d'en ajouter depuis leur onglet respectif. ??
De plus, il a été construit sur un design personnalisé, adaptatif et soigné. \\

Bien qu'ayant rencontré différents problèmes et imprévus, nous nous sommes dépassés afin de les régler. Ainsi, nous avons pu continuer à avancer dans le projet. \\
En entreprise, ces problèmes sont tout autant susceptibles de faire leur apparition. \\
En effet, tout le monde ne travaille pas toujours avec le même environnement, il peut arriver qu'un collègue soit absent plusieurs mois, l'entreprise peut utiliser une technologie peut documentée, \dots \\

Grâce à ce projet de groupe d'un genre inédit sur nos trois années, nous avons appris à travailler en équipe sur un travail tel qu'il pourrait nous être demandé lors de notre stage en entreprise. \\
Cela en fait une corde de plus à ajouter à notre arc.