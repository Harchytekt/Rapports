\section{Choix}
\label{sec:choix}

\subsection{Langage}
\label{sec:langage}

Pour ce projet, notre choix s'est naturellement porté vers le Python.
En effet, en plus d'être le langage de prédilection sur le Raspberry Pi, il dispose d'une
documentation et d'une vaste communauté. \\
En outre, ce langage apporte une facilité pour l'interaction avec les
GPIO \footnote{General-purpose input/output} de la Raspberry Pi et la librairie du
Sense HAT.

\subsection{Raspbian}
\label{sec:raspbian}

Pour notre utilisation, Raspbian, étant le choix suggéré par le constructeur, a été choisi.

\begin{figure}[!h]
  \centering
  \includegraphics[scale=0.06]
  {textures/images/choices/raspbian.pdf}
  \caption{Logo Raspbian}
  \label{fig:raspbian}
\end{figure}

\subsection{Outils de développement}
\label{sec:outils-de-devel}

GNU Emacs et Atom ont été les seuls éditeurs de texte utilisés comme outils de
développement pour leur simplicité ainsi que pour notre familiarité avec ceux-ci.

\begin{figure}[!h]
\centering
\begin{minipage}[c]{0.4\textwidth}
  \centering
  \includegraphics[scale=0.09]
  {textures/images/choices/emacs.pdf}
\caption{Logo Emacs}\label{emacs}
\end{minipage} \qquad
\begin{minipage}[c]{0.4\textwidth}
  \centering
  \includegraphics[scale=0.09]
  {textures/images/choices/atom.pdf}
\caption{Logo Atom}\label{atom}
\end{minipage}
\end{figure}

De plus, nous avons utilisé GitHub qui est un service en ligne permettant
d'héberger notre projet et de ce fait, synchroniser notre travail.

\begin{figure}[!h]
  \centering
  \includegraphics[scale=0.1]
  {textures/images/choices/github.pdf}
  \caption{Logo GitHub}
  \label{fig:github}
\end{figure}

%%% Local Variables:
%%% mode: latex
%%% TeX-master: t
%%% End: